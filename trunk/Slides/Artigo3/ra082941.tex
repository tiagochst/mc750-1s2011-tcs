% -%-%-%-%-%-%-%-%-%-%-%-%-%-%-%-%-%-%-%-%-%-%-%-%-%
% MC750 Interfaces homem-computador               %  
% Reflecting Human Values in the digital Age      %
% Data:10/03/2011                                 %
% Unicamp,Campinas,São Paulo,Brasil               %
% Grupo:                                          %
% - Tiago Chedraoui Silva                         %
% -%-%-%-%-%-%-%-%-%-%-%-%-%-%-%-%-%-%-%-%-%-%-%-%-%
\documentclass[dvips,11pt]{beamer}

%%% fontes %%%
\usepackage[T1]{fontenc}
\usepackage[latin9]{inputenc}
\usepackage[brazil]{babel}    % d� suporte para os termos na língua portuguesa do Brasil


%%% matematicos %%%
\usepackage{amsmath}
\usepackage{amssymb}
\usepackage{mathptmx}

%%% figuras %%%
\usepackage{graphicx}
\usepackage{wrapfig}


%%% tabelas %%%
\usepackage{colortbl}
\usepackage{array}
\usepackage{longtable}
\usepackage{fancyvrb}
\usepackage{color}

%%% outros %%%
\usepackage{url}
\usepackage{textcomp}
\usepackage{hyperref} %internal links
\usepackage{color}       
\usepackage{indentfirst} %retira padrao americano de paragrafos
\usepackage{multicol}    
\numberwithin{table}{section}
\numberwithin{figure}{section} %numercao de figuras por secao


%%% usado para codigo %%%
\usepackage{listings}
\lstset{
  language = Java,
  basicstyle=\footnotesize\ttfamily, 
  numbers=left,               
  numberstyle=\tiny,         
  % stepnumber=2,              
  numbersep=5pt,             
  tabsize=8,                  
  extendedchars=true,  
  breaklines=true,       
  keywordstyle=\color{blue},
  frame=b,         
  stringstyle=\color{red}\ttfamily, 
  showspaces=false,          
  showtabs=false,             
  xleftmargin=17pt,
  framexleftmargin=17pt,
  framexrightmargin=5pt,
  framexbottommargin=4pt,
  % backgroundcolor=\color{lightgray},
  showstringspaces=false              
}
\lstloadlanguages{C}
\usepackage{caption}
\DeclareCaptionFont{white}{\color{white}}
\DeclareCaptionFont{green}{\color{green}}
\DeclareCaptionFormat{listing}{\colorbox[cmyk]{0.43, 0.35, 0.35,0.01}{\parbox{\textwidth}{\hspace{15pt}#1#2#3}}}
% \captionsetup[lstlisting]{format=listing,labelfont=white,textfont=white, singlelinecheck=false, margin=0pt, font={bf,footnotesize}}


%%% extras %%%
\RequirePackage{marvosym} % figuras \Letter \Email 
\usepackage{fancyhdr}     % Headers
\usepackage{epsf}
\usepackage{tikz}
\usetikzlibrary{arrows}
\tikzstyle{block}=[draw opacity=0.7,line width=1.4cm]


%%% Beamer style %%%
\usetheme{Darmstadt}
\usefonttheme[onlylarge]{structurebold} % fontes em negrito
\setbeamerfont*{frametitle}{size=\normalsize,series=\bfseries}%define tamanhos de letras
\setbeamertemplate{navigation symbols}{}% barra de navegacao superior

% \usetheme{Warsaw}      % estilos mais usado =P
% \setbeamercovered{transparent}

% -%-%-%-%-%-%-%-%-%
% Inicio Slides  %
% -%-%-%-%-%-%-%-%-%

%%% CAPA %%%
\title{Design Principles for Visual Communication }                               
% Refletindo os valores humanos na era digital
\subtitle{MC750 - Constru��o de Interfaces Homem-Computador}                               

\author[Tiago S.]{
  Tiago Chedraoui Silva \\
  RA:082941 
}
\institute{Universidade Estadual de Campinas}
\date{\today}
%%% END CAPA %%%

% The main document
\begin{document}

\begin{frame}
  \titlepage
\end{frame}

\begin{frame}{Outline}
  \tableofcontents
\end{frame}


\section{Introdu��o}

\begin{frame}{Introdu��o}
  \begin{itemize}
  \item Atualmente, protocolos web perderam campo para web design devido � uma preocupa��o menos t�cnica e mais efetiva.
  \item A web provocou mudan�as no ensinho, na ind�stria, na pol�tica.
  \item H� mais documentos web que pessoas no mundo.
  \item Ferramentar facilitam cria��o de sistemas distribu�dos e a utiliza��o de algoritmos distribu�dos.  Ex: Hadoop
  \item Computa��o social englova v�rios aspectos da intera��o humana na web (Ex: redes sociais)
    
  \end{itemize}
  
  \pause \begin{alertblock}{Web} %iterative cycle
    
    \begin{itemize}
      \pause  \item  Para CS � um meio de comunica��o, um ve�culo de difus�o de conte�do.
      \pause  \item  Para os novos cursos: � um mecanismo de conte�do din�mico que suporta intera��o social entre v�rios usu�rios. 

    \end{itemize}
  \end{alertblock}

\end{frame}


\begin{frame}{}
  \begin{itemize}
  \item A web precisa ser entendida e estudada como um fen�meno, mas tamb�m como algo a ser desenvolvido para um crescimento futuro e futuras capacidades. 
  \end{itemize}

  \pause \begin{alertblock}{Web} %iterative cycle
    
    \begin{itemize}
      \pause  \item  Micro escala: Web � uma infra estrutura de linguagens artificiais e protocolos.
      \pause  \item  Macro escala: A intera��o dos seres humanos como criar, conectar e consumir informa��o.
    \end{itemize}
  \end{alertblock}

\end{frame}

\end{document}
