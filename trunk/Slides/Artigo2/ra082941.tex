%-%-%-%-%-%-%-%-%-%-%-%-%-%-%-%-%-%-%-%-%-%-%-%-%-%
%  MC750 Interfaces homem-computador              %  
%  Reflecting Human Values in the digital Age     %
%  Data:10/03/2011                                %
%  Unicamp,Campinas,S�o Paulo,Brasil              %
%  Grupo:                                         %
%    - Tiago Chedraoui Silva                      %
%-%-%-%-%-%-%-%-%-%-%-%-%-%-%-%-%-%-%-%-%-%-%-%-%-%
\documentclass[dvips,11pt]{beamer}

%%% fontes %%%
\usepackage[T1]{fontenc}
\usepackage[latin9]{inputenc}
\usepackage[brazil]{babel}    % d� suporte para os termos na l�ngua portuguesa do Brasi


%%% matematicos %%%
\usepackage{amsmath}
\usepackage{amssymb}
\usepackage{mathptmx}

\usepackage{multicol}  
%%% figuras %%%
\usepackage{graphicx}
\usepackage{wrapfig}


%%% tabelas %%%
\usepackage{colortbl}
\usepackage{array}
\usepackage{longtable}
\usepackage{fancyvrb}
\usepackage{color}

%%% outros %%%
\usepackage{url}
\usepackage{textcomp}
\usepackage{hyperref} %internal links
\usepackage{color}       
\usepackage{indentfirst} %retira padrao americano de paragrafos
\usepackage{multicol}    
\numberwithin{table}{section}
\numberwithin{figure}{section} %numercao de figuras por secao


%%% usado para codigo %%%
\usepackage{listings}
\lstset{
  language = Java,
  basicstyle=\footnotesize\ttfamily, 
  numbers=left,               
  numberstyle=\tiny,         
  %stepnumber=2,              
  numbersep=5pt,             
  tabsize=8,                  
  extendedchars=true,  
  breaklines=true,       
  keywordstyle=\color{blue},
  frame=b,         
  stringstyle=\color{red}\ttfamily, 
  showspaces=false,          
  showtabs=false,             
  xleftmargin=17pt,
  framexleftmargin=17pt,
  framexrightmargin=5pt,
  framexbottommargin=4pt,
  %backgroundcolor=\color{lightgray},
  showstringspaces=false              
}
\lstloadlanguages{C}
\usepackage{caption}
\DeclareCaptionFont{white}{\color{white}}
\DeclareCaptionFont{green}{\color{green}}
\DeclareCaptionFormat{listing}{\colorbox[cmyk]{0.43, 0.35, 0.35,0.01}{\parbox{\textwidth}{\hspace{15pt}#1#2#3}}}
%\captionsetup[lstlisting]{format=listing,labelfont=white,textfont=white, singlelinecheck=false, margin=0pt, font={bf,footnotesize}}


%%% extras %%%
\RequirePackage{marvosym} % figuras \Letter \Email 
\usepackage{fancyhdr}     % Headers
\usepackage{epsf}
\usepackage{tikz}
\usetikzlibrary{arrows}
\tikzstyle{block}=[draw opacity=0.7,line width=1.4cm]


%%% Beamer style %%%
\usetheme{Darmstadt}
\usefonttheme[onlylarge]{structurebold}
\setbeamerfont*{frametitle}{size=\normalsize,series=\bfseries}
\setbeamertemplate{navigation symbols}{}

%\usetheme{Warsaw}      % estilos slides
%\usefonttheme[onlylarge]{structurebold}   % fontes em negrito
%\setbeamertemplate{navigation symbols}{}  % barra de navegacao superior
%\setbeamerfont*{frametitle}{size=\normalsize,series=\bfseries} %define tamanhos de letras
%\setbeamertemplate{note page}{plain}      % n�o sei
%\setbeamercovered{transparent}

%-%-%-%-%-%-%-%-%-%
%  Inicio Slides  %
%-%-%-%-%-%-%-%-%-%

%%% CAPA %%%
\title{The Three Paradigms of HCI }                               
\subtitle{MC750 - Constru��o de Interfaces Homem-Computador}                               
 
\author[Tiago S.]{
  Tiago Chedraoui Silva \\
RA:082941 
}
\institute{Universidade Estadual de Campinas}
\date{\today}
%%% END CAPA %%%

% The main document

\begin{document}

\begin{frame}
  \titlepage
\end{frame}

\begin{frame}{Outline}
  \tableofcontents
\end{frame}


\section{Introduction}

\subsection{As cinco transforma��es}


\begin{frame}{Finding pp-partitions of genotype matrices.}
  Now comes the general case:
  \begin{itemize}
  \item The inputs $M$ are \alert{genotype matrices}.
  \item The inputs $M$ \alert{do not allow a perfect phylogeny}.
  \item What is $\chi_{\operatorname{PP}}(M)$?
  \end{itemize}

  % primeira transforma��o
  \begin{exampleblock}{The end of interface stability}
    % what an interface might be, where it ism what allows a user to do
    Os cumputadores s�o cada vez mais parte do nosso ambiente. seja em lugares p�blicos ou privados.
  \end{exampleblock}
  
  % segunda transforma��o
  \begin{exampleblock}{The growth of techno-dependency}
    % vocabulario underpins 
    A computa��o participa quase todos os aspectos da nossa vida
    % Nuvem, internet
    Computadores depende de outros
    % concep��o de computadores: grande gama de tecnologia, sempre ligada/funcionando, sempre conectado em uma teia de sistemas  
  \end{exampleblock}
  
  % terceira transforma��o
  \begin{exampleblock}{The growth in hyperconnectivity}
    \begin{itemize}
    \item Habilidade das novas ferramentas de melhorar a efici�ncia e
      ganhar tempo
    \item<alert@1> Mas, h� um aumento no consumo de tempo devido a presen�a digital
    \end{itemize}
  \end{exampleblock}
  
  % quarta transforma��o
  \begin{block}{The end of the ephemeral}
  \end{block}

  % quinta transforma��o
  \begin{block}{The growth of creative engagement}
  \end{block}
  
\end{frame}


\end{document}

