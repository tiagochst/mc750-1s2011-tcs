%-%-%-%-%-%-%-%-%-%-%-%-%-%-%-%-%-%-%-%-%-%-%-%-%-%
%  MC750 Interfaces homem-computador              %  
%  Template para Slides                           %
%  Data:1/03/2011                                %
%  Unicamp,Campinas,S�o Paulo,Brasil              %
%  Grupo:                                         %
%    - Tiago Chedraoui Silva                      %
%-%-%-%-%-%-%-%-%-%-%-%-%-%-%-%-%-%-%-%-%-%-%-%-%-%
\documentclass[dvips,11pt]{beamer}

%%% fontes %%%
\usepackage[T1]{fontenc}
\usepackage[latin9]{inputenc}
\usepackage[brazil]{babel}    % d� suporte para os termos na l�ngua portuguesa do Brasi


%%% matematicos %%%
\usepackage{amsmath}
\usepackage{amssymb}
\usepackage{mathptmx}

\usepackage{multicol}  
%%% figuras %%%
\usepackage{graphicx}
\usepackage{wrapfig}


%%% tabelas %%%
\usepackage{colortbl}
\usepackage{array}
\usepackage{longtable}
\usepackage{fancyvrb}
\usepackage{color}

%%% outros %%%
\usepackage{url}
\usepackage{textcomp}
\usepackage{hyperref} %internal links
\usepackage{color}       
\usepackage{indentfirst} %retira padrao americano de paragrafos
\usepackage{multicol}    
\numberwithin{table}{section}
\numberwithin{figure}{section} %numercao de figuras por secao


%%% usado para codigo %%%
\usepackage{listings}
\lstset{
  language = Java,
  basicstyle=\footnotesize\ttfamily, 
  numbers=left,               
  numberstyle=\tiny,         
  %stepnumber=2,              
  numbersep=5pt,             
  tabsize=8,                  
  extendedchars=true,  
  breaklines=true,       
  keywordstyle=\color{blue},
  frame=b,         
  stringstyle=\color{red}\ttfamily, 
  showspaces=false,          
  showtabs=false,             
  xleftmargin=17pt,
  framexleftmargin=17pt,
  framexrightmargin=5pt,
  framexbottommargin=4pt,
  %backgroundcolor=\color{lightgray},
  showstringspaces=false              
}
\lstloadlanguages{C}
\usepackage{caption}
\DeclareCaptionFont{white}{\color{white}}
\DeclareCaptionFont{green}{\color{green}}
\DeclareCaptionFormat{listing}{\colorbox[cmyk]{0.43, 0.35, 0.35,0.01}{\parbox{\textwidth}{\hspace{15pt}#1#2#3}}}
%\captionsetup[lstlisting]{format=listing,labelfont=white,textfont=white, singlelinecheck=false, margin=0pt, font={bf,footnotesize}}


%%% extras %%%
\RequirePackage{marvosym} % figuras \Letter \Email 
\usepackage{fancyhdr}     % Headers
\usepackage{epsf}


%%% Beamer style %%%
\usetheme{Warsaw}      % estilos slides
%\usefonttheme[onlylarge]{structurebold}   % fontes em negrito
\setbeamertemplate{navigation symbols}{}  % barra de navegacao superior
\setbeamerfont*{frametitle}{size=\normalsize,series=\bfseries} %define tamanhos de letras
\setbeamertemplate{note page}{plain}      % n�o sei
\setbeamercovered{transparent}

%-%-%-%-%-%-%-%-%-%
%  Inicio Slides  %
%-%-%-%-%-%-%-%-%-%

%%% CAPA %%%
\title{The Three Paradigms of HCI }                               
\subtitle{MC750 - Constru��o de Interfaces Homem-Computador}                               
 
\author[Tiago S.]{
  Tiago Chedraoui Silva \\
RA:082941 
}
\institute{Universidade Estadual de Campinas}
\date{\today}
%%% END CAPA %%%


\begin{document}


%-%-%-%-%-%-%
%  Capa     %
%  Topicos  %
%-%-%-%-%-%-%
\begin{frame}
  \titlepage % slide 1 - capa
\end{frame}

\section{Introdu��o}


\begin{frame}
  \frametitle{Paradigmas}
  \begin{itemize}
    \pause \item O que � um paradigma?
    \begin{itemize}
    \item <1-> Maneira descrever ondas de pesquisa em determinado campo
    \item <1-> Novos paradigmas n�o constestam paradigmas anteriores
    \item <1-> Paradigmas co-existem
    \end{itemize}

    \pause \item Quais elementos de um paradigma em IHC?
    \begin{itemize}
    \item <1-> Um entendimento comum das not�veis propriedades da
      intera��o
   \item <1-> Tipos de quest�es interessantes  sobre as propriedades
    \item <1-> Procedimentos para responder as perguntas
    \item <1-> Conhecimento comum de como interpretar os resultados
      dos procedimentos

    \end{itemize}
  \end{itemize}
\end{frame}

\begin{frame}
  \frametitle{IHC - As ondas}
  \begin{itemize}
  \pause \item Primeira onda 
    \begin{itemize}
       \item <1->Foco no indiv�duo
       \item <1->Fatores humanos
       \item <1->Indiv�duo passivo
    \end{itemize}
    
    \pause \item Segunda onda 
    \begin{itemize}
       \item <3->Foco na comunica��o entre m�quina e homem.
      \item <3-> Atores humanos
       \item <3->Indiv�duo ativo
       \item <3->Foco no grupo

    \end{itemize}
    
    \pause \item Terceira onda
    \begin{itemize}
      \item <4->Tecnologia extrapola fronteira dos locais trabalho
      \item <4->Passa a fazer parte da vida das pessoas
      \item <4-> Novas concep��es: Mobilidade, compartilhamento de dados
    \end{itemize}
  \end{itemize}
\end{frame}
\begin{frame}


  \frametitle{Exemplo - Sistema de alerta em avi�o}
  \begin{itemize}
  \item Primeiro paradigma
    \begin{itemize}
      \pause   \item <1->Pilotos homens 
      \pause \item <1->Id�ia: Usar voz de uma mulher
 
    \end{itemize}
    
    \pause \item Segundo paradigma
    \begin{itemize}
      \pause \item <4->Transmitir informa��o eficientemente
      \pause \item <4->A mesma voz � a melhor para todos os homens e caso?
      \pause \item <4->Classifica��o de tipos de vozes baseado na
      efic�cia de cada uma. 
    \end{itemize}
    
    \pause \item Terceira onda
    \begin{itemize}
      \pause \item <8-> 1� e 2� paradigma -> Cockpit: ambiente masculino.
      \pause \item <8-> E se tiv�ssemos uma mulher?
      \pause \item <8-> 1� e 2� paradigma  falhariam!
      \pause \item <8-> Sistema deve se adaptar ao meio.
    \end{itemize}
  \end{itemize}
\end{frame}

\begin{frame}
  \frametitle{Divis�o}

  \begin{itemize}
    \pause \item Primeiro e Segundo paradigmas
    \begin{itemize}
    \item <2-> Racional
    \item <2-> Trabalho
    \end{itemize}
    
    \pause \item Terceiro paradigma
    \begin{itemize}
    \item <3-> Vida
    \item <3-> Lazer
    \item <3-> Emo��o 
    \end{itemize}
    
  \end{itemize}
\end{frame}

\end{document}
